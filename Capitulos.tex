
\section{Descripción de la Realidad Problemática}

En la actualidad no se cuenta con metodologías practicas para la enseñanza de conceptos básicos de  matemática en estudiantes de nivel secundaria teniendo como consecuencia un pobre entendimiento sobre el tema y en general una vaga concepción de como se pueden aplicar en problemas de la vida real. En el informe nacional de resultados El Perú en PISA 2015\cite{Minedu2017} se muestra claramente la situación. Las Matemáticas representan la base de las ciencias, tener una solida formación en temas del álgebra como los productos notables es importante para el desarrollo de los alumnos en las siguientes etapas de educación y formación académica.

%Actualmente se cuenta con metodologías de enseñanza poco practicas que ayuden a los estudiantes a entender temas de matemáticas como el álgebra en especifico el temas de productos notables\cite{Villagrasa2013}.




\section{Delimitación y Definición del Problema}
    \subsection{Delimitación}
El proyecto tendrá el alcance de poder enseñar productos notables con la metodología de Gamificación usando la realidad aumentada enfocándose en alumnos de nivel secundario.

    \subsection{Definición del Problema}

Las metodologías clásicas de enseñanza en las instituciones educativas son poco eficientes para el aprendizaje de los estudiantes haciendo que en muchos de los casos se pierda el interés del curso y llevar el curso solo por aprobar; esto lleva a los alumnos a poseer una pobre base en las matemáticas.


\section{Formulación del Problema}
¿Como se puede ayudar en el aprendizaje de productos notables a los alumnos de nivel de secundaria?

%las metodologías de enseñanza para aumentar el interés de los estudiantes de nivel secundaria en el área de las matemáticas?
\section{Objetivo de la Investigación}
    \subsection{Objetivo General}

Desarrollar un software que permita el aprendizaje de productos notables a los alumnos de nivel de secundaria con la metodología de gamificación usando la realidad aumentada.

    \subsection{Objetivo Especifico}

        \begin{itemize}
            \item Recopilar y analizar el estado del arte de gamificación, Realidad aumentada, metodología de enseñanza de matemáticas.
            \item Desarrollar una metodología de aprendizaje de acuerdo al estado del arte.
            \item Implementar un software con la metodología propuesta.
            \item Realizar la evaluación del algoritmo propuesto.
        \end{itemize}

\section{Hipótesis de la Investigación}

\section{Variables e Indicadores}

\subsection{Variable Independiente}
\subsubsection{indicadores}
\subsubsection{Índices}
\subsection{Variable Dependiente}
\subsubsection{indicadores}
\subsubsection{Índices}

\section{Viabilidad de la Investigación}
\subsection{Viabilidad Técnica}
Ayudar en el aprendizaje de productos notables a los alumnos de nivel secundaria y aporte a los profesores al introducirlos en nuevas formas de enseñanza.

\subsection{Viabilidad Operativa}
Los alumnos podrán expandir su conocimiento al interactuar con el software que se desarrollara con la metodología de gamificación y la Realidad aumentada así mismo los docentes podrán utilizar el software de forma complementaria a la enseñanza.

\subsection{Viabilidad Económica}

Se cuenta con dispositivos que que soportan las realidad aumentada como celulares, computadora para el desarrollo.

\section{Justificación e Importancia de la investigación}
    \subsection{Justificación}
    El aprendizaje de los conceptos de matemáticas y sus aplicaciones en situaciones reales, en especifico los productos notables es de suma importancia como base en la formación académica. Mantener el interés y la motivación en el alumno constante y de forma natural es lo más adecuado para el aprendizaje\cite{Zarra2001} de manera que se puede obtener su atención desinteresada.
    
    Se desarrollaron Metodologías de aprendizaje en las que involucraban directamente a los alumnos \cite{Villagrasa2013}, estos establecían objetivos y métricas con las que se apoyaban para definir el nivel de éxito o fracaso.

    \subsection{Importancia}


\section{Limitaciones de la Investigación}
Este proyecto se limita a desarrollar un software donde se usara la metodología a proponer para la enseñanza del tema de matemáticas de productos notables. Las pruebas se limitaran a la usabilidad del software por parte de los estudiantes de nivel inicial.

\section{Tipo y Nivel de la Investigación}
\subsection{Tipo de Investigación}
\subsection{Nivel de Investigación}

\section{Método y diseño de la Investigación}
\subsection{Método de Investigación}
\subsection{Diseño de Investigación}

\section{Técnicas e instrumentos de Recolección de la Información}
\subsection{Técnicas}
\subsection{Instrumentos}

\section{Cobertura de Estudio}
\subsection{Universo}
\subsection{Muestra}

\section{Cronograma y Presupuesto}
\subsection{Cronograma}

\begin{table}[H]
\raggedright
\begin{tabularx}{\linewidth}{@{}|p{5cm}|X|X|X|X|X|X|@{}}
\hline
Nombre de la tarea & Jun & Jul & Ago & Sep & Oct& Nov   \\ \hline
Revisión del estado del arte    & x & x &   &   &   &   \\ \hline
Desarrollo de Metodología       &   & x & x & x &   &   \\ \hline
Desarrollo de Software          &   &   & x & x &   &   \\ \hline
Elaboración de Evaluación       &   &   &   & x & x &   \\ \hline
Elaboración de tesis            &   &   &   &   & x & x \\ \hline
Presentación                    &   &   &   &   &   & x \\ \hline
\end{tabularx}
\end{table}


\subsection{Presupuesto}
\begin{enumerate}
    \item Humanos:\\Responsable(Tesista): 1 ejecutor: Hancco Medina Wilder Ivan\\
    Colaboradores: 01 asesor.
    \item Equipos:\\01 Laptop\\01 Smartphone.
    \item Financiamiento:\\Responsable 100\%
\end{enumerate}


\section{Estructura Tentativa del Informe Final}

\bibliography{IEEEabrv,referencias.bib}



